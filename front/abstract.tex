% !TeX root = ../main.tex

\begin{abstract}

本研究探討台灣華語和台灣閩南語間協同變調的感知與產出。實驗發現,與過往北京華語的研究結果不同,台灣華語的協同變調呈現較為對稱的分佈。這樣的分佈在方向性上和台灣閩南語皆為正向;在強度上,不論是隨前變調或是隨後變調在兩個語言皆無異。在兩個語言中,隨前變調皆較隨後變調強烈。本文提出在感知層次處理協同變調的兩種方式:其一乃是藉由較嚴格的聲調界線剔除協同變調後的聲調;此為台灣閩南語所採用的方式。其二是台灣華語所採用的模式:即在音位階段接受協同變調後的聲調為其它聲調,再藉由正常化取得原本的目標聲調。
協同變調在本研究的結果顯示,其應被視為跨語言共性,受到語言特性所左右。此外,本研究的結果也支持正常化乃是語言特有的,而非認知的共性。本研究希望藉由這樣的發現為語言的產出與感知以及語言的共性與特性提供見解。

\end{abstract}

\begin{abstract*}

This study investigates the perception and production of coarticulated tones in Taiwan Mandarin and Taiwan Southern Min. Contrary to previous findings in Beijing Mandarin, Taiwan Mandarin has been found to have a more symmetric distribution of tonal coarticulation. The directionalities of the two effects are assimilatory in both languages, and no linguistic differences of magnitudes were found for both effects. In both languages, the carry-over effects were found to be stronger than the anticipatory effects. Crucially, two perceptual solutions to deal with coarticulated tones are proposed. One is to have 
stricter tone boundaries, and to filter out coarticulated tones at the phonemic level, applied in Taiwan Southern Min; the other, applied in Taiwan Mandarin, is to accept coarticulated tones as lexical tones other than the intended target, and to retrieve the target tone through normaliztion afterwards. Tonal coarticulation seen in this study behaves like a universal constraint, subject to linguistic specificities. In addition, this study supports the idea that normalization is speech-specific, and is not a cognitive generality. This study hopes to shed light on the interaction of production and perception, as well as the issue of language universals and specificities.

\end{abstract*}