% !TeX root = ./main.tex

% --------------------------------------------------
% 資訊設定(Information Configs)
% --------------------------------------------------

\ntusetup{
  university*   = {National Taiwan University},
  university    = {國立臺灣大學},
  college       = {文學院},
  college*      = {College of Liberal Arts},
  institute     = {語言學研究所},
  institute*    = {Graduate Institute of Linguistics},
  title         = {協同變調在台灣華語和台灣閩南語中之產出與感知},
  title*        = {Perception and Production of Coarticulated Tones in Taiwan Mandarin and Taiwan Southern Min},
  author        = {黃柏瑄},
  author*       = {Huang, Bo-Xuan},
  ID            = {R09142003},
  advisor       = {邱振豪},
  advisor*      = {Chenhao Chiu},
  %date          = {2020-05-01},         % 若註解掉,則預設為當天
  %oral-date     = {2020-05-01},         % 若註解掉,則預設為當天
  DOI           = {10.5566/NTU2022XXXXX},
  keywords      = {台灣華語, 台灣閩南語, 協同變調, 聲調感知, 正常化},
  keywords*     = {Taiwan Mandarin, Taiwan Southern Min, tonal coarticulation, tone perception, normalization},
}

% --------------------------------------------------
% 加載套件(Include Packages)
% --------------------------------------------------

%\usepackage[sort&compress]{natbib}      
% 參考文獻
%\usepackage{amsmath, amsthm, amssymb}   
% 數學環境
\usepackage{CJKulem}          % 下劃線、雙下劃線與波浪紋效果
\usepackage[normalem]{ulem}

\usepackage{booktabs}                   % 改善表格設置
\usepackage{multirow}                   % 合併儲存格
\usepackage{diagbox}                    % 插入表格反斜線
\usepackage{array}                      % 調整表格高度
\usepackage{longtable}                  % 支援跨頁長表格
\usepackage{paralist}                   % 列表環境

\usepackage{attachfile}
\usepackage{caption}
\usepackage{subcaption}

\usepackage{lipsum}                     % 英文亂字
\usepackage{zhlipsum}                   % 中文亂字

\usepackage{hhline}

\usepackage{tabularx}

\usepackage{apacite}
\usepackage{natbib}

\usepackage{tikz}
\usetikzlibrary{shapes, arrows}
\tikzstyle{terminator} = [rectangle, draw, text centered, rounded corners, minimum height=2em]
\tikzstyle{connector} = [draw, -latex']

\usepackage{hyperref}
\usepackage{listings}

\definecolor{hypercolor}{rgb}{0.1,0.25,0.2}
\definecolor{codegreen}{rgb}{0,0.6,0}
\definecolor{codegray}{rgb}{0.5,0.5,0.5}
\definecolor{codepurple}{rgb}{0.58,0,0.82}
\definecolor{backcolour}{rgb}{0.95,0.95,0.92}
\lstdefinestyle{mystyle}{
    backgroundcolor=\color{backcolour},   
    commentstyle=\color{codegreen},
    keywordstyle=\color{magenta},
    numberstyle=\tiny\color{codegray},
    stringstyle=\color{codepurple},
    basicstyle=\ttfamily\footnotesize,
    breakatwhitespace=false,         
    breaklines=true,                 
    captionpos=b,                    
    keepspaces=true,                 
    numbers=left,                    
    numbersep=5pt,                  
    showspaces=false,    
    showstringspaces=false,
    showtabs=false,                  
    tabsize=2
}
\lstset{style=mystyle}

\usepackage{tipa}
\newcommand{\tip}{\textipa}

\usepackage{newtxtext,newtxmath}