% !TeX root = ../main.tex

\chapter{Background}\label{chapter:Background}

In this chapter, we shall review past studies on the production of and the normalization for tonal coarticulation in (Beijing) Mandarin and (Taiwan) Southern Min. 
\section{Tonal coarticulation in Mandarin and Southern Min}\label{section:Tonal coarticulation in Mandarin and Southern Min}
In this section, we introduce tonal coarticulation, which refers to the tone variation caused by the preceding/following tones, in Mandarin and Southern Min. Typological findings generally agree on an asymmetric directionality of tonal coarticulation \citep{ChangHsieh2012}, with carry-over effects being strong and assimilatory and anticipatory effects being weak and dissimilatory. This is illustrated in table \ref{table:Typologicaldistribution}.

\begin{flushleft}
\begin{table}[hbt!]
\begin{tabularx}{\textwidth}{l|X|X|}
\hhline{~--}
 & Magnitude & Direction \\
\hhline{~|--}\noalign{\vspace*{\doublerulesep}}
\hhline{-||--}
\multicolumn{1}{|l||}{Carry-over} & Strong & Assimilatory\\
\hhline{-||--}
\multicolumn{1}{|l||}{Anticipatory} & Weak & Dissimilatory\\
\hhline{-||--}
\end{tabularx}
\caption{Typological distribution of tonal coarticulation.}
\label{table:Typologicaldistribution}
\end{table}
\end{flushleft}

While production studies of Mandarin tones seem to support such distribution, both the presence and distribution of tonal coarticulation in (Taiwan) Southern Min is lacking consensus in the literature. For the remainder of this section, we shall go through past studies on the production of tonal coarticulation in these two languages.

\section{Tonal coarticulation in Mandarin}\label{section:Tonal coarticulation in Mandarin}
Following \cite{Chao1968}, several studies have been conducted in attempt to measure tonal coarticulation in Mandarin.

\cite{ShihSproat1992} measured the pitch height of Beijing Mandarin T1 (55)-T2 (35)-T1 (55) and all T1 (55) trisyllabic words in isolation, read sentences, and conversation, and found that the pitch contour flattened as it went from being in isolation to in conversation. The authors found a general positive correlation between the amounts of F0 displacement and the duration of the words. This correlation, however, was weak within the read sentence and conversation groups. The authors additionally looked into quadrisyllabic words where T2 (35) was either in the second (weak) syllable or the third (strong) syllable, and it was found that F0 displacement was lower in weak syllables than in strong syllables. The authors therefore concluded that prosodic stress was the primary reason for Chao's T2-T1 tone variation rule.

A more holistic study on the coarticulation of Beijing Mandarin tones was conducted in \cite{Shen1990}, where trisyllabic non-words /\tip{pa.pa.pa}/ with combinatorial tone combinations were used as test materials. Subjects were asked to pace themselves so the three syllables were of the same durations.  She found that tone variation along with surrounding tones was not exclusive to T2 (35) but was applied to all tone combinations, to varying degrees, and unlike the typological distribution, these effects were symmetric, and affect not only part of, but the whole syllable. It was also noticed that such coarticulatory effect only affects the tone values, but not the direction of the tones. Crucially, T2 (35) was found to raise its adjacent tones the most, followed by T1 (55), while T3 (21) effectively lowered their following tones. T4 (51) also lowered its following tones, but did not raise its preceding tones as mush as T1 (55) or T2 (35).

\cite{Xu1994} additionally investigated Beijing Mandarin rising tone and falling tone in compatible (i.e., low offset before and high onset after the rising tone, and high offset before and low onset after the falling tone) contexts and conflicting (i.e., opposite directions as the compatible contexts) contexts, and found that rising tones were slightly falling in conflicting contexts, and that falling tones were less steep in such contexts. This was rather different from \cite{Shen1990}, where tone movements were invariable and only F0 values were affected.

More results supporting an asymmetric distribution were borne out in \citeauthor{Xu1994a} (\citeyear{Xu1994a}, \citeyear{Xu1997}), where disyllabic syllables of /\tip{ma.ma}/ of tone combinations in carrier sentences with different pitch onsets/offsets were measured. Like \cite{Shen1990}, Xu found coarticulatory effects from both preceding and following tones. What was different was that the anticipatory effects found in \citeauthor{Xu1994a} (\citeyear{Xu1994a}, \citeyear{Xu1997}) were dissimilatory in nature, and the magnitude was also much weaker.

An asymmetry of tonal coarticulation was also found in \cite{LinYan1991}, who also measured quadrisyllabic words, and found that tonal coarticulation in Beijing Mandarin was unidirectional, that is, each tone was under either carry-over or anticipatory effect.

Another inconsistency in the literature was that while \cite{Shen1990} and \citeauthor{Xu1994a} (\citeyear{Xu1994a}, \citeyear{Xu1997}) both found that coarticulatory effects could affect as large as the whole adjacent syllable, \cite{LinYan1991} noted that such effects would fade and did not span across the entirety of the syllable.

Overall, the literature seems to support an asymmetric distribution of (Beijing) Mandarin tonal coarticulation, though inconsistencies exist.

\section{Tonal coarticulation in Taiwan Southern Min}
While past studies on tonal coarticulation in Mandarin are plenty, much fewer related works can be found in (Taiwan) Southern Min. One of them is \cite{Peng1997}, where the syllables /\tip{kaw}/ of different tones followed by a high- (55), mid- (33) or low- (21) level tone in phrase -initial, -medial, and -final and utterance-final positions were used as materials, and produced by 4 (2 females) Taiwan Southern Min native speakers. Both prosodic positions, tonal contexts, and their interactions were found to have significant effects on tone variations. Crucially, the coarticulatory effects (i.e., anticipatory effects) were mostly assimilatory, contrary to what we see in Mandarin and the typological distribution. An identification test also showed that the subjects could detect such coarticulatory effect. In the identification test, the target syllables /\tip{kau}/ were exercised from the original sentences produced by one of the female subjects, and the subjects were asked to guess which syllable the exercised syllable was. The subjects achieved an above-chance identification rate. This was taken by the author as proof that such anticipatory coarticulation was salient enough to be detected by the subjects. It should be noted, however, that  the materials used in these two experiments were sentential, and that one of the main aims of this study was to measure the prosodic effect on tone variation. It is therefore hard to assess to which extent the prosodic effect was at work and how it might have interacted with tonal contexts in sentence production. Crucially, the materials used in the identification test were not synthesized; the duration, prosody, and intensity were all retained. Since tones have intrinsic durations, and other suprasegmental properties may also give hints to the listeners, it seems audacious to assert such results were directly associated with tonal coarticulation.

Two more comprehensive works that studied both the anticipatory and carry-over effects were \cite{Lin1988} and \cite{Wang2002}. In \cite{Lin1988}, Taiwan Southern Min Tones in isolation and in sequence were measured. Though small carry-over effect was observed, the author found generally no much coarticulatory effect in Taiwan Southern Min. This is in contradiction with \cite{Wang2002}, where trisyllabic non-words, with T1' (55)/T7' (21) as the first syllables, T2' (55)/T1' (33)/T7' (21) as the second syllables, and T5 (24)/T2 (51) as the last syllable, were used as test materials. 6 male college students participated in the production. Partial significance of stronger carry-over and weaker anticipatory effects were found between some pairs. It should be noted, however, that the subjects were Taiwanese college students at the 2000s, it is presumable that they were very likely also native in Mandarin, and the words used were non-words, with all the tones except T1' (33) also present in Mandarin. Therefore, there is virtually no knowing to what extent these non-words could be perceived as Taiwan Southern Min by the subjects, and whether their production was influenced by Mandarin. \cite{ShihSproat1992} also noted that Mandarin non-words induced different coarticulatory effects than did real words. It is possible that tonal coarticulation would have different distribution between non-words and real words.

Finally, another related work, \cite{ChangHsieh2012}, which investigated tonal coarticulation in Malaysian Hokkien, a dialect of Southern Min, looked into monosyllables and disyllables. Disyllables were grouped into higher/lower onset/offset groups to compare anticipatory and carry-over effects. Two types of grouping were used: one with the tonal scales, where tone values with 3 or above were categorized as higher, and with 2 or below as lower; the other one used the centroid of the tones as the threshold. The first kind of grouping yielded virtually no significant coarticulatory effect in Malaysian Hokkien, while significant, but mixed effects were found in the second grouping, where both the carry-over and anticipatory effects could be assimilatory or dissimilatory.

Overall, past studies in the literature have rather inconsistent results regarding tonal coarticulation in (Taiwan) Southern Min. This may well be due to the different research goals, test materials, statistical quantifications, and different dialects studied.

\subsection{Section summary}

In this section, we reviewed tonal coarticulation in Mandarin and (Taiwan) Southern Min. Discrepancies exist among the literature. Crucially, while most studies agree tonal coarticulation in Mandarin exists and is asymmetric, a consensus regarding this issue is lacking in (Taiwan) Southern Min. A cross-linguistic comparison will therefore require a consistent and quantifiable experiment design.

\section{Normalization and tonal coarticulation normalization in Mandarin and Taiwan Southern Min}

In this section, we shall first briefly discuss perceptual compensation/normalization in general, including the two major kinds of normalization, and the issue of whether and to what extent normalization is speech-specific. For the second half of this section, we will discuss past studies on the normalization of tonal coarticulations in Mandarin and Taiwan Southern Min.

\subsection{Kinds of normalization and the speech-specificity of normalization}\label{section:Kinds of normalization and the speech-specificity of normalization}

Speech is built on the production and perception of phonetic categories including both segmental and suprasegmental elements. The acoustics of speech in real-life circumstances, however, are highly variable, subject to factors such as talker differences (e.g., the morphology of vocal tracts due to sex, age, and race; \citealp{Markovaetal2016}), acoustic effects of adjacent sounds, or the physical environment in which the speech takes place. These variances can be further categorized into two major types: inter-talker variances, and intra-talker variances \citep{Francisetal2003}. The former includes the formant and pitch range differences caused by the size of vocal tracts, larynxes and morphology of vocal folds \citep{JohnsonSjerps2018}, and individual speech styles. The latter refers to the variances of the same phonological elements within the same speaker. This may be caused by factors such as coarticulation with ambient sounds (\citealp{WangFillmore1961}), the prosodic location of the target element (e.g., \citealp{Peng1997}), or the register of the speaker at the moment of speech (e.g., \citealp{Schaferetal2000}). Tonal coarticulation in this sense is a kind of intra-talker variance.

In order for a speech to be successful, listeners have to make accommodations for such acoustic variances. These accommodations are commonly referred to as ``normalizations'' or ``perceptual compensations'' (cf. \cite{Zhangetal2022} for a more detailed distinction of terminology; in this paper, we follow the conventional ``normalization''. In Section \ref{section:General perceptual compensation vs. speech-specific normalization}, however, we will turn back to this issue, and argue for the term ``normalization''.). Normalization, according to the types of variance, can be inter-talker normalization (or simply talker normalization) or intra-talker normalization (which includes tonal coarticulation normalization). Such theme was first explored in \cite{LadefogedBroadbent1957}, where the authors investigated the perception of four English vowels /\tip{I}/, /\tip{E}/, /\tip{5}/, /\tip{2}/ in the words \textit{bit}, \textit{bet}, \textit{bat}, and \textit{but}, preceded by 6 introduction sentences with varying F1's and/or F2's. Their results showed that acoustically identical tokens were perceived as different words when followed by sentences with different F1-F2 spaces. This suggests that in perception, listeners do not only blindly follow the intrinsic acoustics of the sounds, but make use of extrinsic cues and make normalization accordingly.

Such talker normalization is also found for suprasegmental elements such as the perception of tones. \cite{WongDiehl2003} investigated the tone perception of Cantonese level tones on three monosyllabic words /\tip{si}1/ `teacher', /\tip{si}3/ `to try', and /\tip{si}6/ `yes', with T1, T3, and T6 being respectively high-, mid-, and low- level tones, and found that the identification rates were significantly higher in talker-blocked conditions than when the talkers were mixed within the same block. In addition, such talker normalization was found to change the categorization of level tones when they were preceded by contexts with higher/lower F0's. Contexts 2 semitones higher than the neutral contexts were found to lower the perceived F0 of the target tones, leading all three words to be identified as /\tip{si}6/, and contexts 2 semitones lower raised the perceived F0 of the following words, which were identified as /\tip{si}1/. 

However, several studies have put this kind of speech-specific stance into question, suggesting that what is viewed as normalization is but a general cognitive process that can be induced by even non-speech contexts, which should be more appropriately called ``perceptual compensation''. \citeauthor{WatkinsMakin1994} (\citeyear{WatkinsMakin1994}; \citeyear{WatkinsMakin1996}) imitated the experiment design of \cite{LadefogedBroadbent1957}, but substituted the introduction sentences with non-speech materials, and still successfully induced similar perceptual compensation. 

In Section \ref{section:Intra-talker tone normalization and tonal coarticulation normalization in Taiwan Mandarin and Taiwan Southern Min}, we will look at tone normalization, including intra-talker normalization and normalization for tonal coarticulation, and show that judging from past studies, normalization, at least that of tone, is not purely speech-irrelevant, and thus might be affected by language-specific variables.

\subsection{Intra-talker tone normalization and tonal coarticulation normalization in Taiwan Mandarin and Taiwan Southern Min}\label{section:Intra-talker tone normalization and tonal coarticulation normalization in Taiwan Mandarin and Taiwan Southern Min}

As seen in Section \ref{section:Kinds of normalization and the speech-specificity of normalization}, inter-talker normalization is found not only for segments but also for tones, such as in Cantonese \citep{WongDiehl2003}. This phenomenon is also attested in Mandarin. In \cite{HuangHolt2009}, a T1-T2 continuum, with T1 being a high-level (55) tone and T2 a mid-rising (35) tone as the targets, were preceded by two context sentences, one with an average F0 of 200 Hz, and the other with an average F0 of 165 Hz. They found that higher-F0 context sentences shifted the T1-T2 threshold backward as the targets went from mid-rising to high-level, suggesting that higher-F0 contexts also lowered the perceived F0 of the targets. 

Similar results were borne out in tonal coarticulation normalization in Mandarin. \cite{Xu1994} examined the identification of target T2 (25) and T4 (51) tones with the original preceding tones swapped with higher-/lower- offset tones in trisyllabic non-words. While identification rates were similar in compatible sequences (cf. Section \ref{section:Tonal coarticulation in Mandarin}), the identification rates in conflicting sequences were found to be significantly lower when the tones were swapped. In addition, \cite{Zhangetal2022}, by creating a continuum from T3 (21) to T4 (51), preceded by T1 (55), T2 (35) and T4 (51) with minimal contrasting pairs, also found that higher preceding offsets shifted the perception toward T3 (21), while when the first syllables had lower tone offsets, the targets were more easily perceived as T4 (51).

Related works in Taiwan Southern Min, again are sparse. \cite{Wang2002}, however, did report certain effect of normalization of Taiwan Southern Min tonal coarticulation. The author swapped tones in Taiwan Southern Min trisyllables produced by one of the subjects in the production experiment mentioned in section 1.2, and found that T2' (55) and T1' (33) seemed to be perceived as lower when preceded by higher offsets, and vice versa. The results were not systematic, though. In addition, the author only reported a general significant effect of tone swapping on identification rates. It is hard to tell exactly what kind of effect (i.e., the directionality) it was, and since the measurement was the identification frequency, systematic quantification would be difficult. Crucially, the stimuli were not synthesized, and were non-words. The same issue mentioned in section 1.2 regarding \citeauthor{Peng1997}'s (\citeyear{Peng1997}) identification test design and this author's production test would also exist.

However, if we are to compare normalization of tonal coarticulation between Taiwan Mandarin and Taiwan Southern Min, we shall first address the issue of whether normalization is speech-specific or is simply a general perceptual process. If it is a general perceptual process, immediately there would be no need for such cross-linguistic comparison, since it is not reserved for language in the first place.

The theme of speech-specificity of tone normalization was explored in \cite{HuangHolt2009} and \cite{Zhangetal2022}. In the former, aside from using context sentences of high/low mean F0's, the authors also used pure tones as preceding contexts. Similar results were seen both when the preceding contexts were sentences and when they were non-speech. \cite{Zhangetal2022} also found that non-speech preceding syllables could elicit similar effects as could speech syllables, only to a much lesser degree. \cite{Zhangetal2022} therefore stipulated that tone normalization had both general perceptual processes and other factors involved, which contributed to the larger effect seen for the speech group versus the non-speech group. Why this difference in magnitude was not seen in \cite{HuangHolt2009} could be due to the smaller F0 difference between the context materials \citep{Zhangetal2022}.

Overall, we shall conclude for now that while tone normalization may be partially due to general cognition, speech-specificity, and therefore language-specificity, may also be present. We will return to this issue in Section \ref{section:General perceptual compensation vs. speech-specific normalization}.

\subsection{Section summary}
In this section we reviewed past studies on normalization, inter-talker tone normalization and, more importantly, tonal coarticulation normalization. It is found that both Mandarin and Taiwan Southern Min tone production may be subject to the normalization of coarticulation, though a direct comparison is not possible with the studies at hand.

\section{Summary}
Discrepancies exist in the results of tonal coarticulation and the normalization of it in Mandarin and (Taiwan) Southern Min. Specifically, no study have systematically compared tonal coarticulation in two languages with different tone systems, and a quantifiable test for measuring tonal coarticulation normalization in the two languages, especially Taiwan Southern Min is lacking. This study therefore hopes to fill in this gap with a cross-linguistic comparable design. In the next chapter, we shall talk about the three experiments conducted in this study.