% !TeX root = ../main.tex

\chapter{Results}\label{chapter:Results}

In this section, we will talk about the results of the experiments.

\section{Taiwan Mandarin and Taiwan Southern Min tones under coarticulation}

\subsection{Tone contours}

Tone contours of Taiwan Mandarin and Taiwan Southern Min tones in carryover and anticipatory positions are shown in Appendix \ref{Appendix:ToneContours}.

\subsection{Directionalities and magnitudes of coarticulatory effects in the two languages}

Tonal coarticulation of Taiwan Mandarin and Taiwan Southern Min in carry-over and anticipatory positions respectively are shown in figures \ref{Figure:LMMCarryover} and \ref{Figure:LMMAnticipatory}. In general, tonal coarticulations in the two languages are identical. For both languages, in both positions, the ambient tones were shown to exert positive impacts (both p<.001***) on the target tones, that is, a high ambient tone raised, and a low ambient tone lowered the target tone. This meant that in both languages, both the carry-over and anticipatory effects were assimilatory, and no differences of magnitude of the carry-over effects between the two languages or the anticipatory effects between the two languages were found (p=.31/.65, respectively).

However, difference in magnitudes between the two effects were found in within-language comparisons. In both Taiwan Mandarin and Taiwan Southern Min, the carry-over effects were stronger than the anticipatory effects (p=.01* in Taiwan Mandarin and p=.004** in Taiwan Southern Min) .

No significant differences between monolingual and bilingual group's Taiwan Mandarins were found.

\begin{figure}[hbt!]
\centering
\includegraphics[width=\textwidth, trim={0 .5cm 0 0}]{figures/E1/Carryover_lang_seperated.png}
\caption{Fitted LMM model of tone onsets and offsets in carry-over positions (left: Taiwan Mandarin (monolingual); middle: Taiwan Mandarin (bilingual); right: Taiwan Southern Min).}
\label{Figure:LMMCarryover}
\end{figure}

\begin{figure}[hbt!]
\centering
\includegraphics[width=\textwidth, trim={0 .5cm 0 0}]{figures/E1/Anticipatory_lang_seperated.png}
\caption{Fitted LMM model of tone onsets and offsets in anticipatory positions (left: Taiwan Mandarin (monolingual); middle: Taiwan Mandarin (bilingual); right: Taiwan Southern Min).}
\label{Figure:LMMAnticipatory}
\end{figure}

In general, tonal coarticulations in Taiwan Mandarin and Taiwan Southern Min were identical in both directionality and magnitude. In both languages, both the carry-over effects and anticipatory effects were assimilatory, with the former being stronger, and the later weaker. Rather symmetric patterns were therefore found in the two languages. This can be summarized in table \ref{table:MandarinDistribution}.

\begin{flushleft}
\begin{table}[hbt!]
\begin{tabularx}{\textwidth}{l|X|X|}
\cline{2-3}
 & Magnitude & Direction \\
%\hhline{-::==}
\hhline{~|--}\noalign{\vspace*{\doublerulesep}}
\hhline{-||--}
\multicolumn{1}{|X||}{Carry-over} & Stronger & \multirow{2}{*}{Assimilatory}\\
\hhline{|-||-~}
\multicolumn{1}{|X||}{Anticipatory} & Weaker& \\
\hhline{|-||-|-|}
\end{tabularx}
\caption{Distribution of tonal coarticulation in Taiwan Mandarin and Taiwan Southern Min.}
\label{table:MandarinDistribution}
\end{table}
\end{flushleft}

\section{Normalization for tonal coarticulation in Taiwan Mandarin and Taiwan Southern Min}
Raw falling tone responses of the the monolingual and bilingual groups in Mandarin and Southern Min are shown in Figure \ref{Figure:E2Raw}.

\begin{figure}[hbt!]
\centering
\begin{subfigure}[b]{.45\textwidth}
\centering
\includegraphics[width=\textwidth]{figures/E2/Mandarin_monolingual_E2_raw.png}
\end{subfigure}
\hfill
\begin{subfigure}[b]{.45\textwidth}
\centering
\includegraphics[width=\textwidth]{figures/E2/Mandarin_bilingual_E2_raw.png}
\end{subfigure}
\hfill
\begin{subfigure}[b]{.45\textwidth}
\centering
\includegraphics[width=\textwidth]{figures/E2/Min_E2_raw.png}
\end{subfigure}
\caption{Falling tone response percentages in Experiment 2 (top left: Mandarin (monolingual); top right: Mandarin (bilingual); bottom: Southern Min).}
\label{Figure:E2Raw}
\end{figure}

Upon first sight, we see an obvious difference between the monolingual group's Mandarin results and the bilingual group's Southern Min results, with the latter being generally narrower. As shown in Figure \ref{Figure:E2GAMM}, the distances between the GAMM fitted splines of the monolingual group's Taiwan Mandarin falling tone responses (the maximum distance being 2.78 (level)) were apparently wider than those in Taiwan Southern Min (the maximum distance being 1.53 (level)), and this linguistic difference was seen even within the bilingual group, where the distances between the splines of the bilingual group's Taiwan Mandarin results (the maximum distance being 2.63 (level)) were also wider.

\begin{figure}[hbt!]
\centering
\begin{subfigure}[b]{.45\textwidth}
\centering
\includegraphics[width=\textwidth]{figures/E2/Mandarin_GAMM.png}
\end{subfigure}
\hfill
\begin{subfigure}[b]{.45\textwidth}
\centering
\includegraphics[width=\textwidth]{figures/E2/Mandarin_bilingual_GAMM.png}
\end{subfigure}
\hfill
\begin{subfigure}[b]{.45\textwidth}
\centering
\includegraphics[width=\textwidth]{figures/E2/Min_GAMM.png}
\end{subfigure}
\caption{GAMM fitted falling tone response percentages in Experiment 2 (top left: Mandarin (monolingual); top right: Mandarin (bilingual); bottom: Southern Min).}
\label{Figure:E2GAMM}
\end{figure}

As mentioned in Section \ref{section:Experiment2}, this measurement serves as a means of quantification of the magnitudes of perceptual normalization for tonal coarticulation. The results suggest that normalization for tonal coarticulation was of a smaller amplitude in Taiwan Southern Min than in Taiwan Mandarin, and this difference between Mandarin and Southern Min, interestingly, was present not only between groups, but also within the bilingual subjects.

\section{Tone boundaries between the low tone and the falling tone in Mandarin and Taiwan Southern Min}

Raw acceptance rates of the falling tone and the low tone in Mandarin and Southern Min of the monolingual and bilingual groups are shown in Figure \ref{Figure:E3Raw}.

\begin{figure}[hbt!]
\centering
\begin{subfigure}[b]{.45\textwidth}
\centering
\includegraphics[width=\textwidth]{figures/E3/Mandarin_monolingual_E3_raw.png}
\end{subfigure}
\hfill
\begin{subfigure}[b]{.45\textwidth}
\centering
\includegraphics[width=\textwidth]{figures/E3/Mandarin_bilingual_E3_raw.png}
\end{subfigure}
\hfill
\begin{subfigure}[b]{.45\textwidth}
\centering
\includegraphics[width=\textwidth]{figures/E3/Min_E3_raw.png}
\end{subfigure}
\caption{Acceptance rates in Experiment 3 (top left: Mandarin (monolingual); top right: Mandarin (bilingual); bottom: Southern Min).}
\label{Figure:E3Raw}
\end{figure}

GAMM fitted first derivatives of the acceptance rates are shown in Figure \ref{Figure:E3GAMM}. It is seen that, rather straightforwardly, the falling tone acceptance rates had both higher maximum of first derivatives and higher (i.e., closer to the ideal level of falling tones) threshold levels in Taiwan Southern Min than in Taiwan Mandarin, while for the low tone, such pattern seemed absent. Simple t-tests showed that the falling tone acceptance rates in Taiwan Southern Min had substantially larger maximum of first derivatives of the acceptance rates (p=.019*), and marginally significantly higher levels of thresholds (p=.059). This pattern was more obvious on the advanced bilingual group (p=.005***/.091**, respectively). This pattern, however, was not seen for the low tone, where the maximums of first derivatives of the acceptance rates were higher in Taiwan Mandarin than in Taiwan Southern Min (p=.018*), and the levels of the thresholds were not significantly different (p=.974).

As suggested in Section \ref{section:E3 Analyses}, the maximum of the first derivates of the acceptance rates and the level of threshold were taken as how strict the tone boundary of the target tone was in the language.
\begin{figure}[hbt!]
\centering
\begin{subfigure}[b]{.45\textwidth}
\centering
\includegraphics[width=\textwidth]{figures/E3/Tone21_speed_GAMM.png}
\end{subfigure}
\hfill
\begin{subfigure}[b]{.45\textwidth}
\centering
\includegraphics[width=\textwidth]{figures/E3/Tone51_speed_GAMM.png}
\end{subfigure}
\caption{GAMM fitted first derivatives of acceptance rates in Experiment 3 (left: low tone; right: falling tone). The gray areas indicate intervals of significant differences.}
\label{Figure:E3GAMM}
\end{figure}
The results suggested that the boundary was stricter in Taiwan Southern Min for the falling tone, while for the low tone, this pattern was absent. Nevertheless, we shall argue in section \ref{section:Tone boundaries and perception of tonal coarticulation} that such absence of a stricter boundary for the low tone in Taiwan Southern Min was likely due to tone sandhi rules, and that tone boundaries in general shall be deemed as stricter in Taiwan Southern Min than in Taiwan Mandarin.

As can be seen in Figure \ref{Figure:E3GAMM_bilingual}, again, the linguistic difference was seen within the bilingual subjects. Paired t-tests of within-subject comparisons of the bilingual group showed that for the falling tone, while the maximums of first derivatives of the acceptance rates were not significantly larger in Taiwan Mandarin (p=.158), the threshold was higher (p=.0098***). For the low tone, the maximum of first derivatives of the acceptance rates was substantially larger in Taiwan Mandarin than in Taiwan Southern Min (p=.004**), while the thresholds were not significantly different (p=.620).

\begin{figure}[hbt!]
\centering
\begin{subfigure}[b]{.45\textwidth}
\centering
\includegraphics[width=\textwidth]{figures/E3/Tone21_speed_GAMM_bilingual.png}
\end{subfigure}
\hfill
\begin{subfigure}[b]{.45\textwidth}
\centering
\includegraphics[width=\textwidth]{figures/E3/Tone51_speed_GAMM_bilingual.png}
\end{subfigure}
\caption{GAMM fitted first derivatives of acceptance rates in Experiment 3 (within bilingual groups; left: low tone; right: falling tone). The gray areas indicate intervals of significant differences.}
\label{Figure:E3GAMM_bilingual}
\end{figure}

\section{Summary}
In this section, we have examined tonal coarticulation in Taiwan Mandarin and Taiwan Southern Min, and normalization and tone boundaries in the two languages. It is seen that while Taiwan Mandarin and Taiwan Southern Min exhibited rather similar distribution in terms of tonal coarticulation, Taiwan Southern Min was shown to be less influenced by the normalization effect of tonal coarticulation. Tone boundaries of the falling tone and the low tone in these two languages were also shown to be different in terms of strictness. In the next chapter, we will discuss the significance of these discrepancies and of other results that we have seen in this section.