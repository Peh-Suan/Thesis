% !TeX root = ../main.tex

\chapter{Conclusions}\label{chapter:Conclusions}

This study investigates tonal coarticulation and tone perception under its influence in Taiwan Mandarin and Taiwan Southern Min. It is found that unlike Beijing Mandarin, Taiwan Mandarin, likely under influence of Taiwan Southern Min, has a more symmetric distribution of tonal coarticulation. It is also found that while coarticulatory effects are both strong in the two languages, due to stricter tone boundaries, coarticulated tones are less likely perceived as other lexical tones in Taiwan Southern Min, where the normalizing effects are therefore less strong.

This study supports the argument that tonal coarticulation is a universal constraint, which is subject to other linguistic-specificity, resulting in different distributions among languages. Normalization is once again seen to be speech-specific, strongly affected by the linguistic experiences of individual speakers.

This study is the first study to investigate the issue of tonal coarticulation and related tone perception in Taiwan Mandarin, and to make systematic comparison between the two languages. Specifically, this study discusses a key point crucial to the perception of coarticulated tones, that is, tone boundaries, that has been overlooked in previous studies. The author hopes this study may shed light on the interaction of production and perception, and provide further clues for the issue of linguistic universals and specificities. 
